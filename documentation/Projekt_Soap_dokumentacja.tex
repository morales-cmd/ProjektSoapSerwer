\documentclass[11pt]{article}   % list options between brackets
\usepackage[utf8]{inputenc}
\usepackage[T1]{fontenc}
\usepackage{lmodern}
\usepackage[polish]{babel}
\usepackage{amsmath}
\usepackage{blindtext}
%\usepackage{amsfonts}
%\usepackage{amssymb}
\usepackage{subcaption}
\usepackage{float}
\usepackage{color}
\usepackage{wrapfig}
\usepackage{listings}
\usepackage{graphicx} % Allows including images
%\usepackage{booktabs} % Allows the use of \toprule, \midrule and \bottomrule in tables

% type user-defined commands here

\begin{document}
	\begin{titlepage}
	\title{%
		Projekt Soap Web Service\\
		\large Sklep internetowy z kartami do gry Magic: The Gathering\\
		Specyfikacja Wejście-Wyjście\\
	wersja 1.0}
	\author{Paweł Marczak i Łukasz Kosmaty}         % type author(s) between braces
	\date{\today}    % type date between braces
	\maketitle
	\end{titlepage}
	\tableofcontents
	\section{Wstęp}     % section 1.1
	%\subsection{History}       % subsection 1.1.1
	\subsection{Przeznaczenie}
	Celem zaprojektowanego web service-u jest automatyzacja procesu składania zamówień na karty w sklepie sprzedającym single (karty na sztuki) do gry Magic: The Gathering.\par W obecnej wersji, web service pozwala na przegląd magazynu sklepu (narazie zdefiniowanego statycznie), zalogowanie na istniejące w bazie konto (w tym momencie bez możliwości otwartej rejestacji), uzupełnienie koszyka zamówienia przez użytkownika, złożenie zamówienia (z kontrolą poprawności zamówienia i odpowiednią aktualizacją stanu magazynu) oraz wydrukowanie potwierdzenia zamówienia w formacie pdf. By obslugiwać serwis można korzystać z przygotowanego przez nas klienta obsługiwanego za pośrednictwem html.\par Serwis został stworzony z myślą o dalszym rozwoju. Przykładowymi ścieżkami rozwoju są: integracja z zewnętrznymi bazami danych, możliwość otwartej rejetracji, szyfrowanie przesyłanych komunikatów SOAP (np. przy użyciu standardu SSL), integracja z zewnętrznymi systemami płatniczymi.
	\subsection{Zakres}
Dokument opisuje standardy sieciowe, według których zbudowana jest usługa oraz
prezentuje przykładowy sposób jej używania.
\section{Specyfikacja usługi Web Service}
Usługa ,,Sklep Mtg'' zaimplementowana została jako usługa sieciowa (Web
Service) z użyciem protokołu SOAP. \par
Usługa dostępna jest przez protokół HTTP za pośrednictwem zaprojektowanego w HTML klienta.
	\subsection{Adres usługi}
	
	\subsection{Specyfikacja wsdl}
	 \lstset{
		language=xml,
		tabsize=3,
		%frame=lines,
		caption=Test,
		label=code:sample,
		frame=shadowbox,
		%rulesepcolor=\color{gray},
		xleftmargin=20pt,
		framexleftmargin=15pt,
		keywordstyle=\color{blue}\bf,
		commentstyle=\color{green},
		stringstyle=\color{red},
		numbers=left,
		numberstyle=\tiny,
		numbersep=5pt,
		breaklines=true,
		showstringspaces=false,
		basicstyle=\footnotesize,
		emph={food,name,price},emphstyle={\color{magenta}}}
\lstinputlisting{Mtgsklep_serwerImplService.xml}
\subsection{Operacje usługi}
Usługa zaplanowana jest do obsługi operacji:
\begin{itemize}
	\item zaloguj - zwraca informacje o koncie związanym z podanym loginem i hasłem
	\item zwroc magazyn - zwraca informacje o aktualnym stanie magazynu sklepu
	\item dodajDoKoszyka - ustawia w koszyku klienta podaną liczbę kopii danej karty
	\item usunZKoszyka - usuwa z koszyka daną pozycję (wszystkie karty o danej nazwie)
	\item zlozZamowienie - składa zamówienie (weryfikuje poprawność koszyka, aktualizuje magazyn i stan konta użytkownika, wysyła informacje o potwierdzeniu zamówienia)
\end{itemize}
\subsection{Zwracane obiekty}
Szczegółowiej opisane zostaną tylko elementy budzące wątpliwości.
\subsubsection{Konto}
Obiekty zawierające informacje o zalogowanym użytkowniku, tzn.:
\begin{itemize}
	\item login (String)
	\item haslo (String)
	\item imie (String)
	\item nazwisko (String)
	\item stan\_konta (Float) - wirtualny stan konta służący do realizacji płatności
	\item miasto (String)
	\item kod\_pocztowy (String)
	\item adres (String)
	\item email (String)
	\item numer\_telefonu (String)
	\item ArrayList<Stan> koszyk- lista stanów w koszyku (karta+ liczba kopii karty+cena)
	
\end{itemize}
\subsubsection{Stan}
Obiekty zawierające informacje o danym stanie związanym z daną kartą(zarówno magazyn jak i koszyk klienta składają się z listy stanów ArrayList<Stan>), tzn.:
\begin{itemize}
	\item karta (Karta)- obiekt karta zawierający informacje o karcie
	\item na\_stanie (int) -liczba kopii karty karta w danym stanie
	\item cena (float)- cena pojedynczej kopii karty karta brutto
	\item wartosc\_razem- cena wszystkich kart w danym stanie brutto (cena*na\_stanie)
	
\end{itemize}
\subsubsection{Karta}
Obiekty zawierające informacje o danej karcie, tzn.:
\begin{itemize}
	\item nazwa (String)
	\item opis (String) - krótki opis karty: kolor, Set, rzadkość
	\item ilustracja (Image)- grafika karty

	
\end{itemize}
\subsubsection{Magazyn}
Obiekty zawierające dane o stanie magazynu, tzn.:
\begin{itemize}
	\item ArrayList<Stan> lista\_st - lista stanów, z których złożony jest magazyn
\end{itemize}

\subsubsection{Potw\_zamowienia}
Obiekty zawierające informacje o statusie złożonego zamówienia, tzn.:
\begin{itemize}
	\item kwota (Float)- kwota brutto w pln.
	\item kwota\_netto(Float)- kwota netto w pln.
	\item dane sklepu (Dane\_sklepu)- informacje o sprzedawcy.
	\item czy\_zatwierdzono(Boolean) -True, jeśli zamówienie zostało pomyślnie złożone, False- jeśli nie zostało pomyślnie złożone
	\item message (String)- komentarz dotyczący statusu zamówienia
\end{itemize}
\subsection{Specyfikacja wejście-wyjście opercji Web Service}
By operacje działały poprawnie użytkownicy muszą trzymać się określonych niżej standardów w wysyłanych zapytaniach SOAP.
\subsubsection{Operacja zaloguj}
Dane wejściowe:
\begin{itemize}
	\item login (String)- login skojarzony z kontem
	\item haslo (String)- hasło skojarzone z kontem
\end{itemize}	
Dane wyjściowe:
\begin{itemize}
	\item obiekt typu Konto skojarzony z podanymi loginem i hasłem
	\item zwraca wyjątek, jeżeli w bazie serwera nie ma konta o loginie i haśle podanych w zapytaniu
\end{itemize}
\lstset{
	language=xml,
	tabsize=3,
	%frame=lines,
	caption=Przykładowy SOAP request operacji zaloguj,
	label=code:sample,
	frame=shadowbox,
	%rulesepcolor=\color{gray},
	xleftmargin=20pt,
	framexleftmargin=15pt,
	keywordstyle=\color{blue}\bf,
	commentstyle=\color{green},
	stringstyle=\color{red},
	numbers=left,
	numberstyle=\tiny,
	numbersep=5pt,
	breaklines=true,
	showstringspaces=false,
	basicstyle=\footnotesize,
	emph={food,name,price},emphstyle={\color{magenta}}}
\lstinputlisting{komunikaty_zdjecia/zaloguj_req}

\lstset{
	language=xml,
	tabsize=3,
	%frame=lines,
	caption=Przykładowy SOAP response operacji zaloguj,
	label=code:sample,
	frame=shadowbox,
	%rulesepcolor=\color{gray},
	xleftmargin=20pt,
	framexleftmargin=15pt,
	keywordstyle=\color{blue}\bf,
	commentstyle=\color{green},
	stringstyle=\color{red},
	numbers=left,
	numberstyle=\tiny,
	numbersep=5pt,
	breaklines=true,
	showstringspaces=false,
	basicstyle=\footnotesize,
	emph={food,name,price},emphstyle={\color{magenta}}}
\lstinputlisting{komunikaty_zdjecia/zaloguj_res}
\subsubsection{Operacja zwrocMagazyn}
Dane wejściowe:
\begin{itemize}
	\item brak danych wejściowych
\end{itemize}	
Dane wyjściowe:
\begin{itemize}
	\item obiekt typu Magazyn zgodny z aktualnym stanem magazynu sklepu

\end{itemize}
\lstset{
	language=xml,
	tabsize=3,
	%frame=lines,
	caption=Przykładowy SOAP request operacji zwrocMagazyn,
	label=code:sample,
	frame=shadowbox,
	%rulesepcolor=\color{gray},
	xleftmargin=20pt,
	framexleftmargin=15pt,
	keywordstyle=\color{blue}\bf,
	commentstyle=\color{green},
	stringstyle=\color{red},
	numbers=left,
	numberstyle=\tiny,
	numbersep=5pt,
	breaklines=true,
	showstringspaces=false,
	basicstyle=\footnotesize,
	emph={food,name,price},emphstyle={\color{magenta}}}
\lstinputlisting{komunikaty_zdjecia/zwrocMagazyn_req}

\lstset{
	language=xml,
	tabsize=3,
	%frame=lines,
	caption=Przykładowy SOAP response operacji zwrocMagazyn,
	label=code:sample,
	frame=shadowbox,
	%rulesepcolor=\color{gray},
	xleftmargin=20pt,
	framexleftmargin=15pt,
	keywordstyle=\color{blue}\bf,
	commentstyle=\color{green},
	stringstyle=\color{red},
	numbers=left,
	numberstyle=\tiny,
	numbersep=5pt,
	breaklines=true,
	showstringspaces=false,
	basicstyle=\footnotesize,
	emph={food,name,price},emphstyle={\color{magenta}}}
\lstinputlisting{komunikaty_zdjecia/zwrocMagazyn_res}


\subsubsection{Operacja dodajDoKoszyka}
Operacja wbrew nazwie nie dodaje pozycji do koszyka, a ustawia w koszyku podaną przez klienta pozycję w podanej liczbie sztuk.\newline
Dane wejściowe:
\begin{itemize}
\item login (String)- login skojarzony z kontem, do którego koszyka się odnosimy
\item haslo (String)- hasło skojarzone z kontem, do którego koszyka się odnosimy
\item nazwa\_karty (String)- nazwa karty, którą chcemy włożyć do koszyka (nazwa musi być identyczna z odpowiednią nazwą w magazynie)
\item liczba (int)- liczba kopii karty, którą chcemy ustawić w koszyku
\end{itemize}	
Dane wyjściowe:
\begin{itemize}
	\item obiekt typu Konto ze zaktualizowanym koszykiem
	\item zwraca wyjątek, jeżeli w bazie serwera nie ma konta o loginie i haśle podanych w zapytaniu, lub nie ma karty o podanej nazwie w magazynie.
\end{itemize}
\lstset{
	language=xml,
	tabsize=3,
	%frame=lines,
	caption=Przykładowy SOAP request operacji dodajDoKoszyka,
	label=code:sample,
	frame=shadowbox,
	%rulesepcolor=\color{gray},
	xleftmargin=20pt,
	framexleftmargin=15pt,
	keywordstyle=\color{blue}\bf,
	commentstyle=\color{green},
	stringstyle=\color{red},
	numbers=left,
	numberstyle=\tiny,
	numbersep=5pt,
	breaklines=true,
	showstringspaces=false,
	basicstyle=\footnotesize,
	emph={food,name,price},emphstyle={\color{magenta}}}
\lstinputlisting{komunikaty_zdjecia/dodajDoKoszyka_req}

\lstset{
	language=xml,
	tabsize=3,
	%frame=lines,
	caption=Przykładowy SOAP response operacji dodajDoKoszyka,
	label=code:sample,
	frame=shadowbox,
	%rulesepcolor=\color{gray},
	xleftmargin=20pt,
	framexleftmargin=15pt,
	keywordstyle=\color{blue}\bf,
	commentstyle=\color{green},
	stringstyle=\color{red},
	numbers=left,
	numberstyle=\tiny,
	numbersep=5pt,
	breaklines=true,
	showstringspaces=false,
	basicstyle=\footnotesize,
	emph={food,name,price},emphstyle={\color{magenta}}}
\lstinputlisting{komunikaty_zdjecia/dodajDoKoszyka_res}

\subsubsection{Operacja usunZKoszyka}
Operacja usuwa podaną pozycję z koszyka podanego konta.\newline
Dane wejściowe:
\begin{itemize}
	\item login (String)- login skojarzony z kontem, do którego koszyka się odnosimy
	\item haslo (String)- hasło skojarzone z kontem, do którego koszyka się odnosimy
	\item nazwa\_karty (String)- nazwa karty, którą chcemy usunąć z koszyka (nazwa musi być identyczna z odpowiednią nazwą w magazynie). Usuwa całą pozycję (wsystkie sztuki).

\end{itemize}	
Dane wyjściowe:
\begin{itemize}
	\item obiekt typu Konto ze zaktualizowanym koszykiem
	\item zwraca wyjątek, jeżeli w bazie serwera nie ma konta o loginie i haśle podanych w zapytaniu, lub nie ma karty o podanej nazwie w koszyku.
\end{itemize}
\lstset{
	language=xml,
	tabsize=3,
	%frame=lines,
	caption=Przykładowy SOAP request operacji usunZKoszyka,
	label=code:sample,
	frame=shadowbox,
	%rulesepcolor=\color{gray},
	xleftmargin=20pt,
	framexleftmargin=15pt,
	keywordstyle=\color{blue}\bf,
	commentstyle=\color{green},
	stringstyle=\color{red},
	numbers=left,
	numberstyle=\tiny,
	numbersep=5pt,
	breaklines=true,
	showstringspaces=false,
	basicstyle=\footnotesize,
	emph={food,name,price},emphstyle={\color{magenta}}}
\lstinputlisting{komunikaty_zdjecia/usunZKoszyka_req}

\lstset{
	language=xml,
	tabsize=3,
	%frame=lines,
	caption=Przykładowy SOAP response operacji usunZKoszyka,
	label=code:sample,
	frame=shadowbox,
	%rulesepcolor=\color{gray},
	xleftmargin=20pt,
	framexleftmargin=15pt,
	keywordstyle=\color{blue}\bf,
	commentstyle=\color{green},
	stringstyle=\color{red},
	numbers=left,
	numberstyle=\tiny,
	numbersep=5pt,
	breaklines=true,
	showstringspaces=false,
	basicstyle=\footnotesize,
	emph={food,name,price},emphstyle={\color{magenta}}}
\lstinputlisting{komunikaty_zdjecia/usunZKoszyka_res}
\subsubsection{Operacja zlozZamowienie}
Operacja sprawdza poprawność koszyka (spójność z magazynem), stan konta użytkownika i zwraca potwierdzenie transakcji oraz aktualizuje stan magazynu i opróżnia koszyk.\newline
Dane wejściowe:
\begin{itemize}
	\item login (String)- login skojarzony z kontem, do którego koszyka się odnosimy
	\item haslo (String)- hasło skojarzone z kontem, do którego koszyka się odnosimy
	
\end{itemize}	
Dane wyjściowe:
\begin{itemize}
	\item obiekt typu Potw\_zamowienia z wszelkimi informacjami dotyczącymi transakcji
	\item zwraca wyjątek, jeżeli w bazie serwera nie ma konta o loginie i haśle podanych w zapytaniu.
	\item jeżeli zamówienie nie zostało potwierdzone, wtedy w potwierdzeniu zwracane jest czy\_zatwierdzono = False i komentarz w polu message.
\end{itemize}
\lstset{
	language=xml,
	tabsize=3,
	%frame=lines,
	caption=Przykładowy SOAP request operacji zlozZamowienie,
	label=code:sample,
	frame=shadowbox,
	%rulesepcolor=\color{gray},
	xleftmargin=20pt,
	framexleftmargin=15pt,
	keywordstyle=\color{blue}\bf,
	commentstyle=\color{green},
	stringstyle=\color{red},
	numbers=left,
	numberstyle=\tiny,
	numbersep=5pt,
	breaklines=true,
	showstringspaces=false,
	basicstyle=\footnotesize,
	emph={food,name,price},emphstyle={\color{magenta}}}
\lstinputlisting{komunikaty_zdjecia/zloz_zamowienie_req}

\lstset{
	language=xml,
	tabsize=3,
	%frame=lines,
	caption=Przykładowy SOAP response operacji zlozZamowienie,
	label=code:sample,
	frame=shadowbox,
	%rulesepcolor=\color{gray},
	xleftmargin=20pt,
	framexleftmargin=15pt,
	keywordstyle=\color{blue}\bf,
	commentstyle=\color{green},
	stringstyle=\color{red},
	numbers=left,
	numberstyle=\tiny,
	numbersep=5pt,
	breaklines=true,
	showstringspaces=false,
	basicstyle=\footnotesize,
	emph={food,name,price},emphstyle={\color{magenta}}}
\lstinputlisting{komunikaty_zdjecia/zloz_zamowienie_res}

\end{document}
