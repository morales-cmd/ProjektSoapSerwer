\documentclass[11pt]{article}   % list options between brackets
\usepackage[utf8]{inputenc}
\usepackage[T1]{fontenc}
\usepackage{lmodern}
\usepackage[polish]{babel}
\usepackage{amsmath}
\usepackage{blindtext}
%\usepackage{amsfonts}
%\usepackage{amssymb}
\usepackage{subcaption}
\usepackage{float}
\usepackage{wrapfig}
\usepackage{graphicx} % Allows including images
%\usepackage{booktabs} % Allows the use of \toprule, \midrule and \bottomrule in tables

% type user-defined commands here

\begin{document}
	\begin{titlepage}
	\title{%
		Projekt Soap Web Service\\
		\large Sklep internetowy z kartami do gry Magic: The Gathering\\
		Specyfikacja Wejście-Wyjście\\
	wersja 1.0}
	\author{Paweł Marczak i Łukasz Kosmaty}         % type author(s) between braces
	\date{\today}    % type date between braces
	\maketitle
	\end{titlepage}
	\tableofcontents
	\section{Wstęp}     % section 1.1
	%\subsection{History}       % subsection 1.1.1
	\subsection{Przeznaczenie}
	Celem zaprojektowanego web service-u jest automatyzacja procesu składania zamówień na karty w sklepie sprzedającym single (karty na sztuki) do gry Magic: The Gathering.\par W obecnej wersji, web service pozwala na przegląd magazynu sklepu (narazie zdefiniowanego statycznie), zalogowanie na istniejące w bazie konto (w tym momencie bez możliwości otwartej rejestacji), uzupełnienie koszyka zamówienia przez użytkownika, złożenie zamówienia (z kontrolą poprawności zamówienia i odpowiednią aktualizacją stanu magazynu) oraz wydrukowanie potwierdzenia zamówienia w formacie pdf. By obslugiwać serwis można korzystać z przygotowanego przez nas klienta obsługiwanego za pośrednictwem html.\par Serwis został stworzony z myślą o dalszym rozwoju. Przykładowymi ścieżkami rozwoju są: integracja z zewnętrznymi bazami danych, możliwość otwartej rejetracji, szyfrowanie przesyłanych komunikatów SOAP (np. przy użyciu standardu SSL), integracja z zewnętrznymi systemami płatniczymi.
	\subsection{Zakres}
Dokument opisuje standardy sieciowe, według których zbudowana jest usługa oraz
prezentuje przykładowy sposób jej używania.
\section{Specyfikacja usługi Web Service}

	



\end{document}
